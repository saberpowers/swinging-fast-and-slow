\documentclass[10pt]{article}

\usepackage[alldates=year, block=space, style=apa]{biblatex}
\setlength\bibitemsep{0.5\baselineskip}
\addbibresource{../swingfastslow.bib}

\usepackage{mathptmx}
\usepackage[margin=1in]{geometry} 
\setlength\parindent{0pt}
%\RequirePackage[authoryear]{natbib}
\RequirePackage[colorlinks,citecolor=blue,urlcolor=blue]{hyperref}%% uncomment this for coloring bibliography 
\RequirePackage{graphicx}%% uncomment this for including figures

\begin{document}
 
\title{Response to Reviewers}
\date{}

\maketitle

We thank the Associate Editor and the reviewer for their critical assessment of our paper.
Below, we provide a point-by-point response to their comments.
AE/reviewer comments are in normal font and our responses are bold-faced. In our revised manuscript, all changes are highlighted in red.

\section*{Associate Editor}

Thank you. A reviewer has made numerous suggestions as to revising this piece, which I agree would help TAS readers get into it in more detail. I hope you will consider those revisions seriously. 

\bigskip

\textbf{We thank...}

\section*{Reviewer}

The authors have written a manuscript that studies the impact of recently released bat tracking metrics (bat speed and swing length) on contact and power outcomes. They use a multi-stage approach that includes Bayesian hierarchical models, instrumental variables regression, and Markov chains. They conclude that reducing bat speed as the number of strikes increase can reduce a hitters strikeout rate at the cost of power. The paper is well-written, includes a great amount of detail, and will likely appeal to those familiar with baseball analytics. I do not have any major criticisms of the analytic approach, but do believe there are some opportunities to make the approach more robust as well as opportunities to improve clarity for a more general readership.

\bigskip

\textit{Major Criticisms}

\begin{enumerate}
    \item Lack of Clarity in the Narrative: Those familiar with baseball analytics (and sports analytics in general) will probably be able to follow the methodology. However, there are a number of moving parts (and “building blocks” as the authors call them) and I am concerned that general readers of The American Statistician might not easily put the pieces together. There are two areas that need improvement:
    \begin{itemize}
        \item A visualization akin to a consort diagram that outlines how the sample was reduced. In various places, the authors discuss necessary reductions to the sample size (full swings, squared up contact) but a flow-chart that walks through the initial sample size down to the size for analysis would improve clarity.
        \item A similar flow chart that shows the modeling progression (Bayesian model $\rightarrow$ causal model $\rightarrow$ Markov chain) and where the various building blocks (linear weights, hit outcome, pitch outcome) feed into the system.
    \end{itemize}
\end{enumerate}

\textbf{We thank the reviewer ...}

\bigskip

\textit{Minor Criticisms}

\begin{enumerate}
    \item I would actually categorize these criticisms as somewhere between minor and major, and they pertain to the analytic approach:
    \begin{enumerate}
        \item Did the authors consider fitting a multivariate hierarchical model? Figure 6 seems to reconcile random effects from the separate models, but I’m wondering whether a more unified modeling approach would make a difference.
        \item Are there any measures of global model fit that could be reported? I’m fully convinced that the skew-normal models were more appropriate than Gaussian, but how much of the variance of bat speed and swing length are explained by the skew-normal model? Similar question for the causal model.
        \item Is there a reason why the handedness of the batter and pitcher was not accounted for in the hierarchical model? Since some hitters perform so much worse against particular handed pitchers, I wonder whether that would be an important source of variation to control for. The authors hint at further exploration of batter profiles as future work, but controlling for handedness is reasonable given that they already have this data (unless there is a plausible theoretical reason for not doing so!).

        \bigskip
        \textbf{We...}
        
    \end{enumerate}
    \item Page 2: The authors reference the open research question regarding the effect of batter aggressiveness. On what? Winning? Individual performance? All of the above?
    \item Page 3: It is a bit unconventional to provide a result (Figure 1) prior to the results section. I understand the intent but wonder whether this could be avoided or discussed at a high level until the nature of the data set is conveyed to the readers.
    \item Page 8: The authors reference video review – is this something they did, or is it referencing something? Please elaborate a little more.
    \item Page 9: Why was the 2022-2024 date range chosen?
    \item I was a little bit confused by the notation in equation 2 (though it may be due to my confusion about the model itself). We are told that b\_i represents swing i against pitcher p\_i. Should pitchers have their own index? What is the sample space for i?
    \item The use of names as points in some of the figures muddled the interpretation some. I think there is value to labeling edge cases, but I wonder whether it would be more effective to use points and then selectively label the edge cases.
    \item I was surprised to see Treat Turner at the bottom of Table 6. His WAR values in 2022-24 were 5.2, 3.5, and 3 which runs contrary to how we should interpret Table 6. Is there something contextual to this? Mediocre OBP? Lineup order?
\end{enumerate}




\printbibliography


 
\end{document}